\documentclass{article}
\usepackage[cm]{fullpage}
\usepackage[vlined,boxed,commentsnumbered]{algorithm2e}
\usepackage{multicol}
\usepackage{url}
\usepackage{amsmath,amsthm,amssymb}
\usepackage{graphicx}
\usepackage{listings}
\usepackage{fancyvrb}
\usepackage{multirow}

\newcommand{\codigofuente}[1]{
	\lstinputlisting[language=Java]{#1}
}

\newcommand{\seccion}[2]{
\section{#1}					% #1 es el nombre de la seccion
	\input{#2}					% #2 es la ruta del archivo
}

\lstset{ %
language=Java,                % the language of the code
basicstyle=\footnotesize,       % the size of the fonts that are used for the code
}

\newcommand{\imagen}[2]{
\begin{center}
	\includegraphics[width=#1]{#2}
\end{center}
}



\newcommand{\algoritmo}[1]{
\begin{algorithm}
	\DontPrintSemicolon
	\LinesNumbered
	\SetFuncSty{textsc}

	\SetKwInOut{Input}{Entrada}\SetKwInOut{Output}{Salida}
	\SetKw{KwDownTo}{down to}\SetKw{KwTo}{to}
	
	\input{#1}
\end{algorithm}
}

\renewcommand{\refname}{Bibliograf\'ia:}
\renewcommand{\contentsname}{Tabla de contenidos:}
\SetAlgorithmName{Algoritmo}{Lista de algoritmos}

\setlength{\columnsep}{0.25in}
\setlength{\columnseprule}{1px}

\author{}
\title{Notebook de maratones \\
\includegraphics[width=30mm]{logo.png} }

\begin{document}

\maketitle

\tableofcontents
\begin{itemize}
	\item La instrucci\'on de asignaci\'on se denota por el s\'imbolo $\leftarrow$.
	\item El intercambio de dos variables (\textsl{swap} en ingl\'es) se denota por el s\'imbolo $\leftrightarrow$.
	\item Los \'indices de los arreglos inician en 1. Es decir si $L$ es un arreglo, entonces su primer elemento es se\~nalado por $L[1]$ y en consecuencia su \'ultimo elemento ser\'a $L[$\textsc{Length}$(L)]$.
\end{itemize}

\pagebreak

\seccion{Temas b\'asicos}{./basics_and_format/basics_and_format}
\seccion{Dividir y conquistar}{./divide&conquer/divide&conquer}
\seccion{Estructuras de datos}{./ds/ds}
\seccion{Teor\'ia de n\'umeros}{./number_theory/number_theory}
\seccion{Programaci\'on din\'amica}{./dp/dp}          
\seccion{Grafos}{./graph_theory/graph_theory}
\seccion{Geometr\'ia computacional}{./geometry/geometry}
\seccion{\textit{String matching}}{./stringology/stringology}
\seccion{Combinatoria}{./combinatory/combinatory}
\seccion{Probabilidad y estad\'istica}{./probability/probability}
\seccion{\'Algebra lineal}{./linear_algebra/linear_algebra}
\seccion{Otros}{./others/others}

\pagebreak

\begin{thebibliography}{9}

\bibitem{introductionToAlgorithms}
  Thomas H. Cormen, Charles E. Leiserson, Ronald L. Rivest, Stein Clifford,
  \textit{Introduction to Algorithms},
  Segunda edici\'on. MIT Press, 2003.

\bibitem{algorithmsAndComplexity}
  Herbert S. Wilf,
  \textit{Algorithms and complexity},
  Segunda edici\'on. A K Peters , 2002.

\bibitem{algorithmicGraphTheory}
  David Joyner, Minh Van Nguyen, Nathan Cohen,
  \textit{Algorithmic Graph Theory},
  Versi\'on 0.3. Recuperado el 30 de marzo de 2010 de \url{http://code.google.com/p/graph-theory-algorithms-book/}

\end{thebibliography}

\end{document}
