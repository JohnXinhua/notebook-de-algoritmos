\begin{multicols}{2}

\textbf{N\'umero de factores de un n\'umero:}

La factorizaci\'on prima de un n\'umero $n$ contiene a lo sumo log$_2n$ factores.\\
\textbf{Peque\~no Teorema de Fermat:}

Si $p$ es un primo y $a$ es un n\'umero natural positivo primo relativo con $p$ entonces $a^{p-1} \equiv 1$ mod $p$.\\
\textbf{Teorema chino del residuo:}

Sean $m_1,m_2,\cdots m_n$ enteros de los cuales todos son primos relativos entre s\'i. Adem\'as sean $a_1,a_2, \cdots a_n$ enteros arbitrarios. Entonces el sistema:
\begin{center}
$x\equiv a_1\mod m_1$\\
$x\equiv a_2\mod m_2$\\
$ \vdots$\\
$x\equiv a_n\mod m_n$
\end{center}
tiene una \'unica soluci\'on m\'odulo $m=m_1m_2\cdots m_n$\\
\textbf{Funci\'on Phi de Euler:}

La funci\'on $\phi (n)$ denota el n\'umero de enteros en $\{1,2,\cdots ,n\}$ que son primos relativos a $n$. 
\begin{itemize}
\item Si $a$ y $b$  son primos relativos entonces $\phi(ab)=\phi(a)\phi(b)$
\item Si $p$ es un primo entonces $\phi(p^k)=p^k-p^{k-1}$.
\end{itemize} \\

\textbf{N\'umero de primos hasta $n$:}

Sea $\pi(n)$ el n\'umero de primos entre $1,2,\cdots ,n$. Entonces $\pi(n)\sim \frac{n}{\ln(n)}$

\end{multicols}
\subsection{Algoritmo euclidiano}
	\algoritmo{./number_theory/al-euclid}
\pagebreak
\subsection{Algoritmo euclidiano extendido}
	\algoritmo{./number_theory/al-extended-euclid}
\subsection{Como resolver ecuaciones lineales modulares}
	\algoritmo{./number_theory/al-modular-eq-solver}
\subsection{Algoritmo de exponenciaci\'on modular (Computar $b^n\mod m$)}
	\algoritmo{./number_theory/al-mod-exp}
\pagebreak
\subsection{Como verificar si un n\'umero es primo}
Algunos hechos:
\begin{itemize}
	\item Todos los primos mayores que $3$ pueden escribirse en la forma $6k+/-1$
	\item Cualquier n\'umero $n$ puede tener solamente un factor primo mayor a $\sqrt{n}$
\end{itemize}
	\algoritmo{./number_theory/al-is-prime}
\subsection{Conseguir los divisores de los primeros $n$ n\'umeros}
	\codigofuente{./number_theory/divisores_primeros_n_numeros.java}