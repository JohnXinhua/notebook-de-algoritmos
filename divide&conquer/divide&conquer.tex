\begin{multicols}{2}
\subsection{Pasos de un algoritmo de dividir y conquistar}
\begin{enumerate}
	\item \textbf{Dividir:} Dividir el problema en instancias m\'as peque\~{n}as del mismo problema.
	\item \textbf{Conquistar:} Resolver recursivamente cada subproblema.
	\item \textbf{Combinar:} Juntar las respuestas parciales
\end{enumerate}
Un algoritmo de este tipo tendr\'a una complejidad:
\begin{center} $T(n)=aT(n/b)+f(n)$ \end{center}
donde $a$ es el n\'umero de problemas m\'as peque\~{n}os que se resuelven recursivamente, $b$ es la fracci\'on del problema original que se resuelve en un suproblema y $f(n)$ es lo que cuesta combinar las respuestas.
\subsection{El teorema maestro}
Sean $a\geq 1$ y $b>1$ constantes, sea $f(n)$ una funci\'on y sea $T(n)$ una funci\'on en los enteros no negativos definida por la recurrencia:
\begin{center} $T(n)=aT(n/b)+f(n)$ \end{center}
donde $n/b$ puede interpretarse como $\lfloor n/b \rfloor$ o $\lceil n/b \rceil$. Entonces $T(n)$ puede ser acotada asint\'oticamente as\'i:
\begin{enumerate}
	\item Si $f(n)= O(n^{log_ba-\epsilon})$ para alguna constante $\epsilon>0$ entonces \\$T(n)=\Theta(n^{log_ba})$.
	\item Si $f(n)= \Theta(n^{log_ba})$ entonces $T(n)=\Theta(n^{log_ba}$Lg$(n))$.
	\item Si $f(n)= \Omega(n^{log_ba+\epsilon})$ para alguna constante $\epsilon>0$ y si $af(n/b)\leq cf(n)$ para alguna constante $c<1$ y un $n$ lo suficientemente grande, entonces $T(n)=\Theta(f(n))$.
\end{enumerate}
\subsection{N\'umeros Fibonacci}
Aprovechando la identidad 
\[
\begin{pmatrix}
  F_{n+1} & F_n \\
  F_n & F_{n-1}
\end{pmatrix}
 =
\begin{pmatrix}
  1 & 1 \\
  1 & 0
\end{pmatrix}^n
\]
y el algoritmo de dividir y conquistar para exponenciaci\'on se puede obtener el $n$-\'esimo n\'umero fibonacci en tiempo logaritmico.
\subsection{Algoritmo de Strassen (Multiplicaci\'on de matrices)}
Para obtener el producto de dos matrices $AB=C$, donde $A,B$ y $C$ son matrices $n\times n$, asumiendo que $n$ es una potencia exacta de $2$, cada matriz se puede dividir en cuatro matrices $n/2\times n/2$. Entonces la ecuaci\'on $AB=C$ se puede reescribir como:
\[
\begin{pmatrix}
  a & b \\
  c & d
\end{pmatrix}
\begin{pmatrix}
  e & f \\
  g & h
\end{pmatrix}
=
\begin{pmatrix}
  r & s \\
  t & u
\end{pmatrix}
\]
\begin{align*}
P_1 = & a(f-h) & P_5 = & (a+d)(e+h) \\
P_2 = & (a+b)h & P_6 = & (b-d)(g+h) \\
P_3 = & (c+d)e & P_7 = & (a-c)(e+f) \\
P_4 = & d(g-e) & \\ \\
r = & P_5+P_4-P_2+P_6 & t = & P_3+P_4 \\
s = & P_1+P_2 & u = & P_5+P_1-P_3-P_7
\end{align*}
Se realizan $7$ multiplicaciones de matrices $n/2\times n/2$. La suma de matrices toma tiempo proporcional a $n^2$. Por lo tanto tenemos que $T(n)=7T(n/2)+\Theta(n^2)=\Theta(n^{lg7})$
\end{multicols}
\pagebreak
\subsection{Algoritmos de ordenamiento}
	\subsubsection{Mergesort}
		\algoritmo{./divide&conquer/al-merge}
		\algoritmo{./divide&conquer/al-mergesort}
	\pagebreak
	\subsubsection{Quicksort}
		\algoritmo{./divide&conquer/al-split}
		\algoritmo{./divide&conquer/al-quicksort}
\pagebreak
\subsection{Encontrar en tiempo lineal el $i$-\'esimo menor elemento de un conjunto}
	El algoritmo \textsc{Randomized-Partition} es el mismo \textsc{Split} que utiliza \textsc{Quicksort}.
	\algoritmo{./divide&conquer/al-randomized-select}